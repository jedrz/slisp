\documentclass[a4paper,11pt]{article}
\usepackage[MeX]{polski}
\usepackage[utf8]{inputenc}
\usepackage{indentfirst}
\usepackage[pdftex]{graphicx}
\usepackage{hyperref}
% \hypersetup{colorlinks=true}
\usepackage{amsmath}
\usepackage{color}
\usepackage[usenames]{xcolor}
\frenchspacing
% Wstawki tak jak w pliku źródłowym.
% \usepackage[section]{placeins} (niekoniecznie)
\usepackage[]{float}
\restylefloat{figure}
\usepackage{longtable}
\usepackage{minted}
\usepackage{forest}
\usepackage[shellescape,latex]{gmp}

\newcommand{\clj}[1]{
\inputminted[fontsize=\footnotesize,frame=single,samepage=true]{clojure}{code/#1.clj}
}

\newcommand{\impl}[1]{
\inputminted[fontsize=\footnotesize,frame=single,samepage=true]{java}{impl/#1.java}
}

\author{Łukasz Jędrzejewski}
\title{Techniki kompilacji -- projekt wstępny}
\date{}

\begin{document}

\maketitle

\section{Wstęp}

Projekt dotyczy napisania interpretera prostego języka podobnego do~wielu
dialektów LISP-a~w~języku Java 8.

\section{Konstrukcje języka}

\subsection{Gramatyka języka}

\begin{verbatim}
program = { form }
form = atom | list | vector | quote_form
quote_form = ' form
list = '(' { form } ')'
vector = '[' { form } ']'
atom = symbol | string | number
symbol = [a-zA-Z+-*/<>=%]+
string = "?.*"
number = [0-9]*\.?[0-9]+

symbol_form = symbol form
set! = '(' 'set!' { symbol_form } ')'
fn = '(' 'fn' '[' { symbol } ']' form { form } ')'
defn = '(' 'defn' symbol '[' { symbol } ']' form { form } ')'
if = '(' 'if' form form form { form } ')'
while = '(' 'while' form { form } ')'
do = '(' 'do' form { form } ')'
let = '(' 'let' '[' { symbol_form } ']' form { form } ')'
\end{verbatim}
%prec_form = ('\'' | '`' | '~' | '@' | '~@') form|

\subsection{Ewaluacja form}

Obliczenie wyrażenia listowego polega na:
\begin{enumerate}
\item obliczenie pierwszego elementu, którego wartość powinna być funkcją lub
  specjalną formą,
\item obliczenie argumentów czyli reszty elementów wyrażenia w~przypadku
  funkcji, lub nie w~przypadku specjalnej formy,
\item wywołanie funkcji z~tak przygotowanymi wartościami.
\end{enumerate}

Na~przykład \verb|(+ 1 (- a))|:
\begin{enumerate}
\item obliczenie \verb|+|, czyli zwrócenie globalnej funkcji \verb|+|
\item obliczenie 1, czyli wynikiem będzie 1,
\item obliczenie \verb+(- a)+:
  \begin{enumerate}
  \item obliczenie -,
  \item obliczenie \verb+a+, czyli wartość symbolu, np. 5,
  \item wywołanie funkcji - z~argumentem 5, wynik -5
  \end{enumerate}
\item wywołanie funkcji \verb|+| z~argumentami: 1 oraz -5, wynik -4.
\end{enumerate}

Warto zaznaczyć, że~każde wywołanie zawsze coś zwraca, np. wywołanie \verb+if+
zwraca wynik drugiego lub ostatniego wyrażenia.

Dodatkowo żadna funkcja za~wyjątkiem \verb+set!+, nie modyfikuje wartości
przekazywanych argumentów.

\subsection{Typy}

\begin{description}
\item[symbol] czyli zmienna z~pewną wartościa,
\item[liczba] liczba zmiennoprzecinkowa,
\item[napis]
\item[funkcja]
\item[lista] np., \verb+(list a 1 "napis")+ zwróci listę zawierającą wartość
  symbolu \verb+a+, liczbę 1 oraz napis. Dodatkowo listę można utworzyć
  poprzez wykorzystanie funkcji \verb+quote+, która zapobiega obliczeniu
  podanej listy w~argumencie, np. \verb+(quote (a 1 "napis"))+ zwróci
  \verb+(a 1 "napis")+, symbol \verb+a+ nie zostanie obliczony. Równoważne
  wywołaniu \verb+quote+ jest użycie apostrofu w~następujący sposób:
  \verb+'(a 1 "napis")+.
\end{description}

Dodatkowo symbol \verb+true+ jest zastrzeżony i~traktowany jako prawda,
a~symbol \verb+false+ oraz \verb+nil+ jako fałsz. Jedynie symbole \verb+false+
oraz \verb+nil+ mają wartość logiczną fałsz. Stąd 0, pusta lista, pusty napis
nie będą fałszywe w~sensie logicznym.

\subsection{Specjalne formy}

Czyli ,,funkcje'', których wywołanie nie powoduje obliczania argumentów.

\begin{description}
\item[' quote] nie oblicza wyrażenia.
  \clj{quote}

%\item[inny-apostrof `] nie oblicza wyrażenia, ale pozwala na~obliczanie
%  oznaczonych wyrażeń.
%
%\item[tylda \textasciitilde] oblicza wyrażenie w~ramach wyrażenia poprzedzonego
%  \verb+`+.
%  \clj{unquote}
%
%\item[tylda+małpa \textasciitilde@] dodaje listę do~zacytowanej
%  \clj{unquote-splicing}

\item[set!] przypisanie.
  \clj{set!}

\item[defn] definicja funkcji. Użycie \verb+&+ oznacza, że~pozostałe argumenty
  zostaną przekazane jako lista.
  \clj{defn}

\item[fn] funkcja anonimowa.

\item[do] wykonuje kolejno podane wyrażenia i~zwraca wynik ostatniego.

\item[if] zwraca wynik wykonania drugiego wyrażenia, jeśli pierwsze wyrażenie
  oblicza się do~prawdy, lub ostatniego, wykonując kolejno wyrażenia od~3
  do~ostatniego.
  \clj{if}

\item[when] zwraca wynik wykonania drugiego wyrażenia, jeśli pierwsze wyrażenie
  oblicza się jako prawda lub \verb+nil+ w~przeciwnym wypadku.
  \clj{when}

\item[while] wykonuje wyrażenia od~drugiego do~ostatniego dopóki zachodzi
  warunek podany jako pierwszy argument.
  \clj{while}

\item[let] deklaracja zmiennych lokalnych.
  \clj{let}

\item[eval] oblicza podane wyrażenie.
  \clj{eval}
\end{description}

\subsection{Prymitywy}

\begin{description}
\item[equals] porównuje podane argumenty.

\item[and, or, not] zwraca odpowiednio iloczyn, sumę oraz negację logiczną
  podanych argumentów.

\item[list?, symbol?, number?, string?, function?] sprawdza czy podany argument
  jest pewnego typu.

\item[load] ładuje plik z~kodem.
\end{description}

\subsubsection{Funkcje wyższego rzędu}

Czyli funkcje przyjmujące funkcje jako argumenty.

\begin{description}
\item[map] dla każdego elementu listy wywołuje podaną funkcję i~zwraca listę
  zmapowanych elementów.
  \clj{map}

\item[filter] zwraca listę zawierającą elementy spełniający podany predykat.
  \clj{filter}

\item[reduce] wywołuje funkcję z~akumulatorem i~elementem listy, której wynik
  staje się nowym akumulatorem. Jeśli trzeci argument jest dany, wtedy
  w~pierwszym kroku akumulator jest równy temu argumentowi. W~przeciwnym
  wypadku akumulatorem staje się pierwszy element listy, a~przekazywanym
  elementem drugi.
  \clj{reduce}

\item[apply] wywołuje podaną funkcję z~argumentami zawartymi w~podanej liście.
  \clj{apply}

\end{description}

\subsubsection{Operacje na~liczbach}

\begin{description}
\item[+ - * / \%] podstawowe funkcje matematyczne przyjmujące dowolną ilość
  argumentów (za~wyjątkiem \%).

\item[$> \ >= \ = \ <= \ <$] predykaty porównujące podane liczby (możliwie
  więcej niż 2).

\item[zero?] czy liczba jest równa 0.
\end{description}

\subsubsection{Operacje na~listach i~napisach}

\begin{description}
\item[count] zwraca długość napisu lub listy.
\end{description}

\subsubsection{Operacje na~listach}

\begin{description}
\item[first] zwraca pierwszy element listy.

\item[rest] zwraca listę bez pierwszego elementu.

\item[list] zwraca listę złożoną z~podanych argumentów.

\item[cons] zwraca listę z~dodanym pierwszym argumentem na~początek listy.

\item[conj] zwraca listę z~dodanym drugim argumentem na~koniec listy.

\item[concat] łączy podane listy.

\item[nth] zwraca element o~podanym indeksie.

\item[empty?] czy lista jest pusta.

\item[sort] sortuje listę wg~podanego predykatu.
  \clj{sort}

\end{description}

\subsubsection{Operacje na~napisach}

\begin{description}
\item[str] zamienia podane argumenty na~napis.

\item[subs] zwraca podciąg o~podanym zakresie.
  \clj{subs}

\item[char-at] zwraca literę na~podanej pozycji.
  \clj{char-at}
\end{description}

\subsubsection{Wejście-wyjście}

\begin{description}
\item[print, println] drukuje podane argumenty.
\item[read] wczytuje linię i~ją~zwraca.
\end{description}

\section{Wymagania}

\subsection{Funkcjonalne}

\begin{enumerate}
\item Interpreter powinien pozwolić na~ewaluację form korzystających
  z~powyższych funkcji umieszczonych w~pliku lub podawanych przez użytkownika
  w~trybie konwersacyjnym.
\item W~trybie konwersacyjnym wynik obliczenia formy powinien być wypisany
  w~następnej linii.
\item Podczas wywołania funkcji, wszystkie funkcje, które są~używane
  w~wywoływanej funkcji powinny być zdefiniowane. Jeśli nie są, użytkownik
  zostanie powiadomiony o~niezdefiniowanym symbolu.
\item Dopuszczalne jest użycie niezdefiniowanych funkcji w~definicji funkcji.
\item Nadmiarowe nawiasy są~dopuszczalne, jednak użytkownik zostanie
  poinformowany o~tym zajściu.
\item Brakujące nawiasy spowodują wystąpienie błędu.
\item W~przypadku błędu w~ciele funkcji, błąd zostanie zgłoszony dopiero
  podczas jej użycia.
\item Niepoprawne użycie funkcji (zła liczba argumentów, niepoprawne typy,
  itd.) powinno zostać zasygnalizowane.
\end{enumerate}

\subsection{Niefunkcjonalne}

\begin{enumerate}
\item Testy zostaną stworzone z~wykorzystaniem biblioteki \verb+JUnit+.
\end{enumerate}

\section{Analizator leksykalny}

Zadaniem analizatora leksykalnego jest rozpoznawanie następujących tokenów
w~strumieniu wejściowym:

\begin{itemize}
\item symbol,
\item napis,
\item liczba,
\item nawias otwierający,
\item nawias zamykający,
\item nawias kwadratowy otwierający,
\item nawias kwadratowy zamykający,
\item apostrof.
\end{itemize}

Ponadto analizator leksykalny powinien zgłosić niepoprawne tokeny, np.
\verb+0.0.0+.

Napotkanie znaku \verb+;+ oznacza początek komentarza, więc znaki po~nim
występujące są~ignorowane do~końca linii.

Interfejs analizatora leksykalnego:
\impl{Lexer}

oraz klasa \verb+Token+:
\impl{Token}

\section{Analizator składniowy}

Tworzy drzewo odpowiadające podanemu wyrażeniu przetwarzając kolejno tokeny.

Kiedy napotka token oznaczający:
\begin{description}
\item[symbol] zwraca obiekt opisujący symbol,
\item[liczbę] zamienia napis na~liczbę zmiennoprzecinkową i~ją~zwraca,
\item[napis] zwraca obiekt opisujący napis,
\item[nawias otwierający] tworzy nową listę, do~której dodaje sparsowane tokeny
  do~momentu napotkania nawiasu zamykającego, zwraca tak zbudowaną listę. Jeśli
  nie napotka nawiasu zamykającego, sygnalizowany jest błąd składni,
\item[nawias zamykający] powinien być przetworzony przy obsłudze nawiasu
  otwierającego, oznacza niepotrzebny nawias zamykający,
\item[nawiasy kwadratowe] postępuje analogicznie jak w~przypadku nawiasów
  okrągłych.
\item[apostrof] rozwija w~\verb+(quote ...)+.
\end{description}

\subsection{Przykładowe drzewo rozbioru}

Dla wyrażenia:
\clj{fact}

powstanie drzewo:

% \Tree[.LIST \textit{defn}
%             [.VEC  !\qsetw{-5cm} \textit{n} ]
%             [.LIST \textit{if}
%                    [.LIST \textit{$<=$}
%                           \textit{n}
%                           \textit{1} ]
%                    \textit{1}
%                    [.LIST \textit{*}
%                           [.LIST \textit{fact}
%                                  [.LIST \textit{-}
%                                         \textit{n}
%                                         \textit{1} ] ]
%                           \textit{n} ] ] ]

\resizebox{\linewidth}{!}{
\begin{forest}
[LIST [SYM [defn]]
      [SYM [fact]]
      [VEC [n]]
      [LIST [SYM [if]]
            [LIST [SYM [{$<=$}]]
                  [SYM [n]]
                  [NUM [1]]]
            [NUM [1]]
            [LIST [SYM [*]]
                  [SYM [n]]
                  [LIST [SYM [fact]]
                        [LIST [SYM [-]]
                              [SYM [n]]
                              [NUM [1]]]]]]]
\end{forest}
}

Interfejs analizatora składniowego:
\impl{Parser}

\section{Interpreter}

Interpreter powinien przyjmować wyrażenie do~ewaluacji w~postaci drzewa
będącego wynikiem działania analizator składniowego i~zwracać wynik jego
wykonania w~ramach globalnego zakresu.

W~konstruktorze interpretera najpierw ładowane są~do~tablicy symboli
(globalnego zakresu) napisane w~Javie prymitywy i~funkcje specjalne,
a~następnie funkcje napisane już w~implementowanym języku.

\subsection{Zakres}

Zakres, czyli mapa symboli i~wartości im~przypisanym. Na~początku interpreter
ma~zdefiniowany globalny zakres, w~którym są~umieszczone wbudowane prymitywy
oraz specjalne formy. Oprócz tego dodane zostają funkcje napisane
w~implementowanym języku.

Każdy zakres lokalny, utworzony np. podczas deklaracji zmiennych lokalnych
przechowuje odnośnik do~nadrzędnego zakresu.

Wyszukiwanie podanego symbolu polega na~odszukaniu go~w~zakresie. Jeśli symbol
nie zostaje znaleziony, wtedy należy spróbować znaleźć go~w~nadrzędnym
zakresie. Jeśli poszukiwanie się nie powiodło, oznacza to, że~symbol jest
niezdefiniowany.

Interfejs:
\impl{Scope}

\subsection{Metoda eval interpretera}

Metoda \verb+eval+ interpretera wykonuje podane drzewo (obiekt
\verb+LispObject+) wywołując na~nim metodę \verb+eval+ w~ramach globalnego
zakresu.

Szkielet interpretera:
\impl{Interpreter}

\subsection{Klasy implementujące interfejs LispObject}

Klasy implementujące interfejs \verb+LispObject+ (czyli metodę \verb+eval+) to
\verb+Num+, \verb+Str+, \verb+Bool+, \verb+Nil+, na~rzecz których wywowałanie
\verb+eval+ zwraca sam obiekt.

W~inny sposób zachowują się metody \verb+eval+ symbolu i~listy.

\subsubsection{Obliczenie symbolu}

Obliczenie symbolu polega na~zwrócenia wartości, które jest do~niego
podczepiona w~podanym zakresie, lub jeśli takowa nie istnieje poinformowaniu,
że~symbol jest niezdefiniowany.

\subsubsection{Obliczenie listy}

Obliczenie listy, tak jak wcześniej zostało opisane, polega na~obliczeniu
pierwszego elementu, który powinien implementować interfejs \verb+Callable+
(metodę \verb+call+, która dla podanej listy argumentów oraz zakresu powinna
zwrócić wynik wywołania). Reszta elementów listy to~argumenty, które zostają
przekazane podczas wywołania \verb+call+. Argumenty \textbf{nie są} obliczane.

Przed wywołaniem \verb+call+, następuje sprawdzenie argumentów (ich liczby,
typów, itd.).

Szkielet klasy \verb+Lst+:
\impl{Lst}

\subsection{Klasy implementujące interfejs Callable}

Klasy, które implementują interfejs \verb+Callable+ (metodę \verb+call+
i~\verb+validate+), to~\verb+Primitive+, \verb+SpecialForm+, \verb+Function+
oraz \verb+Macro+.

\subsubsection{Wywołanie prymitywu}

Interfejs prymitywu (wbudowanej funkcji):
\impl{Primitive}
Przekazane argumenty przed właściwym wykonaniem są~obliczane.

Przykładowa implementacja:
\impl{ConcatPrimitive}

\subsubsection{Wywołanie specjalnej formy}

Interfejs specjalnej formy (wbudowanej funkcji, do~której przekazywane
argumenty nie są~obliczane) to~po~prostu rozszerzenie interfejsu
\verb+Callable+.

Przykładowa implementacja formy \verb+if+:
\impl{IfForm}
oraz formy \verb+fn+, zwracającej funkcję anonimową zdefiniowaną przez
użytkownika:
\impl{FnForm}

\subsubsection{Wywołanie funkcji}

Funkcja zdefiniowana za~pomocą \verb+fn+ lub \verb+defn+ jest reprezentowana
jako obiekt zawierający listę nazw argumentów (symboli), ciało funkcji
w~postaci jednego wyrażenia oraz zakres, w~którym funkcja została zdefiniowana.

Szkielet klasy \verb+Function+:
\impl{Function}

Wywołanie zdefiniowanej przez użytkownika funkcji wymaga stworzenia nowego
zakresu, którego zakresem nadrzędnym będzie zakres funkcji rozszerzony
o~symbole, czyli nazwy argumentów w~definicji funkcji oraz odpowiadające
im~wartości przekazane w~wywołaniu. Następnie jest wywoływana metoda
\verb+eval+ na~rzecz ciała funkcji w~celu obliczenia ciała funkcji w~otoczeniu
wcześniej zbudowanego zakresu.

\subsubsection{Wywołanie makra}

Makro zdefiniowane za~pomocą wywołania \verb+defmacro+ powinno zwracać kod
do~wykonania. Argumenty podczas rozwijania makra \textbf{nie są} obliczane.

Szkielet klasy \verb+Macro+:
\impl{Macro}

Mechanizm makr został zrealizowany w~prymitywny sposób. Wykonanie wywołania
makra polega na~rozwinięciu makra, a~następnie wykonaniu wyniku tego
rozwinięcia.

\subsection{Sprawdzenie poprawności użycia funkcji}

Przed właściwym wywołaniem funkcji, najpierw wołana jest metoda
\verb+validate+, która sprawdza poprawność przekazanych argumentów (liczbę
przekazanych argumentów, ich typy, itd.).


%\subsection{Diagram klas typów w~języku}
%
%\begin{mpost}[use,mpsettings={input metauml;}]
%Interface.LispObject("LispObject")();
%ClassName.Bool("Bool");
%ClassName.Function("Function");
%ClassName.Lst("Lst");
%ClassName.Nil("Nil");
%ClassName.Num("Num");
%ClassName.Primitive("Primitive");
%ClassName.SpecialForm("SpecialForm");
%ClassName.Str("Str");
%ClassName.Sym("Sym");
%ClassName.Vec("Vec");
%
%topToBottom(50)(LispObject, Vec);
%leftToRight(7)(Bool, Sym, Num, Str, Lst, Vec, Function, Primitive, SpecialForm);
%topToBottom(50)(Bool, Nil);
%
%drawObjects(LispObject, Bool, Sym, Num, Str, Lst, Vec, Function, Primitive,
%SpecialForm, Nil);
%
%link(realization)(Bool.n -- LispObject.s);
%link(realization)(Function.n -- LispObject.s);
%link(realization)(Lst.n -- LispObject.s);
%link(realization)(Num.n -- LispObject.s);
%link(realization)(Primitive.n -- LispObject.s);
%link(realization)(SpecialForm.n -- LispObject.s);
%link(realization)(Str.n -- LispObject.s);
%link(realization)(Sym.n -- LispObject.s);
%link(realization)(Vec.n -- LispObject.s);
%
%link(inheritance)(Nil.n -- Bool.s);
%\end{mpost}

\end{document}
