\documentclass[a4paper,11pt]{article}
\usepackage[MeX]{polski}
\usepackage[utf8]{inputenc}
\usepackage{indentfirst}
\usepackage[pdftex]{graphicx}
\usepackage{hyperref}
% \hypersetup{colorlinks=true}
\usepackage{amsmath}
\usepackage{color}
\usepackage[usenames]{xcolor}
\frenchspacing
% Wstawki tak jak w pliku źródłowym.
% \usepackage[section]{placeins} (niekoniecznie)
\usepackage[]{float}
\restylefloat{figure}
\usepackage{longtable}
\usepackage{minted}


\newcommand{\clj}[1]{\inputminted[fontsize=\footnotesize]{clojure}{code/#1.clj}}


\author{Łukasz Jędrzejewski}
\title{Techniki kompilacji -- projekt wstępny}
\date{}

\begin{document}

\maketitle

\section{Wstęp}

Projekt dotyczy napisania interpretera prostego języka podobnego do~wielu
dialektów LISP-a~w~języku Java 8.

\section{Konstrukcje języka}



\subsection{Specjalne formy}

Czyli ,,funkcje'', których wywołanie nie powoduje obliczania argumentów.

\begin{description}
\item[quote] nie oblicza wyrażenia.
  \clj{quote}

\item[set!] przypisanie.
  \clj{set!}

\item[def] definicja zmiennej.
  \clj{def}

\item[defn] definicja funkcji.
  \clj{defn}

\item[fn] funkcja anonimowa.

\item[do] wykonuje kolejno podane wyrażenia i~zwraca wynik ostatniego.

\item[if] zwraca wynik wykonania drugiego wyrażenia, jeśli pierwsze wyrażenie
  oblicza się do~prawdy, lub ostatniego, wykonując kolejno wyrażenia od~3
  do~ostatniego.
  \clj{if}

\item[while] wykonuje wyrażenia od~drugiego do~ostatniego o~ile zachodzi
  warunek podany jako pierwszy argument.
  \clj{while}

\item[let] deklaracja zmiennych lokalnych.
  \clj{let}
\end{description}

\subsection{Prymitywy}

\begin{description}
\item[+ - * /] podstawowe funkcje matematyczne przyjmujące dowolną ilość
  argumentów.

\item[> < = >= <=] predykaty porównujące liczby.

\item[equals] porównuje podane argumenty.

\item[and, or] zwraca odpowiednio iloczyn oraz sumę logiczną podanych
  argumentów.

\item[help] zwraca dokumentację symbolu.

\item[load] ładuje plik z~kodem.
\end{description}

\subsubsection{Funkcje wyższego rzędu}

Czyli funkcje przyjmujące funkcje jako argumenty.

\begin{description}
\item[map] dla każdego elementu listy wywołuje podaną funkcję i~zwraca listę
  zmapowanych elementów.
  \clj{map}

\item[filter] zwraca listę zawierającą elementy spełniający podany predykat.
  \clj{filter}

\item[reduce] wywołuje funkcję z~akumulatorem i~elementem listy, której wynik
  jest staje się nowym akumulatorem. Jeśli trzeci argument jest dany, wtedy
  w~pierwszym kroku akumulator jest równy temu argumentowi. W przeciwnym
  wypadku akumulatorem staje się pierwszy element listy, a przekazywanym
  elementem drugi.
  \clj{reduce}

\item[apply] wywołuje podaną funkcję z~argumentami zawarty w~podanej liście.
  \clj{apply}

\end{description}

\subsubsection{Operacje na~listach}

\begin{description}
\item[first] zwraca pierwszy element listy.

\item[rest] zwraca listę bez pierwszego elementu.

\item[list] zwraca listę złożoną z~podanych argumentów.

\item[cons] zwraca listę z~dodanym pierwszym argumentem na~początek listy.

\item[conj] zwraca listę z~dodanym drugim argumentem na~koniec listy.

\item[nth] zwraca element o~podanym indeksie.

\item[empty?] czy lista jest pusta.

\end{description}

\subsubsection{Operacje na~napisach}

\begin{description}
\item[length] zwraca długość napisu.

\item[concat] zwraca napis będący konkatenacją podanych.

\item[subs] zwraca podciąg o~podanym zakresie.
  \clj{subs}
\end{description}

\subsubsection{Wejście-wyjście}

\begin{description}
\item[print] drukuje napis.
\item[read] wczytuje linię i~ją~zwraca.
\end{description}

\section{Analizator leksykalny}

\section{Analizator składniowy}

\section{Interpreter}


\end{document}
